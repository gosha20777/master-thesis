\subsection{Предлагаемый подход}\label{sect-2-1}

Настоящим представляется конвейер обработки изображений на основе AW-Net, а также описывается метод применения доменной адаптации к данному конвейеру. Схема нейросетевой архитектуры данного конвейера представлена на рисунке \ref{main-arch}. Учитывая, что выходным результатом является RGB-изображение, а входным - RAW-изображение, данная нейросеть должна научиться учитывать специфические особенности аппаратного обеспечения камеры в дополнение к изучению встроенного в устройство ISP конвейера.

\addimghere{2-1-main-arch}{0.9}{Схема нейросетевой архитектуры.}{main-arch}

Стоит отметить, что разрыв домена имеет место, поскольку модель, разработанная с использованием данных с одной камеры (исходный домен), не работает аналогичным образом при применении к данным с другой камеры (целевой домен) в силу уникальных особенностей каждого устройства. Особенности сенсоров и ISP конвейера обычно различаются, при этом ISP является проприетарным программным обеспечением, что делает его реверс-инжиниринг практически невозможным. Для решения данных ограничений, предлагается метод адаптации домена, позволяющий обучить конвейер обработки на небольшом наборе данных для получения выходных результатов в целевом домене.

\addimghere{2-1-main-arch-da}{0.9}{Иллюстрация предлагаемого подхода к обучению ISP-конвейеров с этапами предобучения и адаптации домена.}{main-arch-da}

В дальнейшем, исходный набор данных обозначается как достаточно большой для обучения нейросетевого конвейера с хорошим качеством, а набор данных для целевого домена считается небольшим (около 10 изображений). Цель данного исследования заключается в адаптации нейронного ISP конвейера, обученного на исходном домене, к целевому домену без заметной потери качества. Для достижения этой цели, модель обучается генерировать изображения RGB с использованием как исходного, так и целевого доменов в качестве входных данных. Описываемый метод проиллюстрирован на рисунке \ref{main-arch-da}. Это сквозная обучаемая глубокая сеть, которая принимает изображение RAW в качестве входных данных и выполняет реконструкцию изображения, используя малый набор помеченных данных для адаптации исходного домена к целевому.

\subsubsection{Архитектура конвейера}\label{sect-2-1-1}

Архитектура рассматреваемого решения представляет собой AW-Net с уменьшеным по сравнению с оригиналом числом блоков повышающей и понижающей дискретизации, к которой были присоеденены дополнительные модули, необходимые для доменной адаптации конвейера, таие как: класификатор домена и преэнкодеры. Стоит также отметить, что подобная модификация может быть применима и к другим U-Net подобным ISP конвеерам, что, несомненно, значительно расширяет границы применимости рассматриваемого подхода. Рассмотрим каждый из вышеописанных блоков более детально.

\input{inc/2-1-1-1-pre-encoders} % Преэнкодер
\input{inc/2-1-1-2-common-aw-net} % AW-Net
\input{inc/2-1-1-3-domain-classifyer} % AW-Net
 % Архитектура конвейера
\subsubsection{Алгоритм обучения}\label{sect-2-1-2}

Для того чтобы обеспечить успешную доменную адаптацию конвейера обработки изображений, предлагаетсяпровести обучение сети в несколько этапов, которые помогут добиться наилучших результатов.

\begin{enumerate}
    \item Первый этап заключается в предварительном обучении конвейера в исходном домене. На данном этапе используется только прекодер исходного домена и выход RGB AW-Net. При этом выходы классификатора домена не учитываются, а преэнкодер целевого домена не используется. При выполнении данного этапа конвейер обработки изображений обучается на большом наборе данных как если бы это был end-2-end конвейер без доменной адаптации.
    \item Второй этап заключается в инициализации целевого преэнкодера весами из исходного преэнкодера. Это позволяет начать обучение сети на основе уже настроенных параметров, что может значительно ускорить процесс обучения.
    \item Третий этап представляет собой этап адаптации домена, где производится обучение всй сети, используя весь исходный домен и небольшую часть целевого домена (10 изображений). На этом этапе на каждом шаге обучения производится последовательная подача оответвующих пар изображений RAW-RGB из целевого и исходного доменов в соответствующий преэнкодер и вычисляются соответствующие функции потерь. При этом учитываются предсказания классификатора предметной области и его обратный градиент.
\end{enumerate}

Такой подход позволяет обеспечить эффективную адаптацию домена, учитывая индивидуальные особенности каждого домена и выделяя доменно-инвариатные признаки. Кроме того, предварительное обучение и инициализация целевого преэнкодера позволяют снизить вычислительную сложность и ускорить процесс обучения а также значительно сократить число размеченных данных, что является важным преимуществом данного подхода. Стоит также отметить, как показано в эксериментах, что используя предлагаемый подход, удается достичь сопоставимого с конвейором обученным с нуля на полном наборе данных качества формирования изображений.

 % Алгоритм обучения
\input{inc/2-1-3-losses} % Losses