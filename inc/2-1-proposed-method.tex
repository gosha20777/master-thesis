\subsection{Предлагаемый подход}\label{sect-2-1}

Настоящим представляется конвейер обработки изображений на основе AW-Net, а также описывается метод применения доменной адаптации к данному конвейеру. Схема нейросетевой архитектуры данного конвейера представлена на рисунке \ref{main-arch}. Учитывая, что выходным результатом является RGB-изображение, а входным - RAW-изображение, данная нейросеть должна научиться учитывать специфические особенности аппаратного обеспечения камеры в дополнение к изучению встроенного в устройство ISP конвейера.

\addimghere{2-1-main-arch}{0.9}{Схема нейросетевой архитектуры.}{main-arch}

Стоит отметить, что разрыв домена имеет место, поскольку модель, разработанная с использованием данных с одной камеры (исходный домен), не работает аналогичным образом при применении к данным с другой камеры (целевой домен) в силу уникальных особенностей каждого устройства. Особенности сенсоров и ISP конвейера обычно различаются, при этом ISP является проприетарным программным обеспечением, что делает его реверс-инжиниринг практически невозможным. Для решения данных ограничений, предлагается метод адаптации домена, позволяющий обучить конвейер обработки на небольшом наборе данных для получения выходных результатов в целевом домене.

\addimghere{2-1-main-arch-da}{0.9}{Иллюстрация предлагаемого подхода к обучению ISP-конвейеров с этапами предобучения и адаптации домена.}{main-arch-da}

В дальнейшем, исходный набор данных обозначается как достаточно большой для обучения нейросетевого конвейера с хорошим качеством, а набор данных для целевого домена считается небольшим (около 10 изображений). Цель данного исследования заключается в адаптации нейронного ISP конвейера, обученного на исходном домене, к целевому домену без заметной потери качества. Для достижения этой цели, модель обучается генерировать изображения RGB с использованием как исходного, так и целевого доменов в качестве входных данных. Описываемый метод проиллюстрирован на рисунке \ref{main-arch-da}. Это сквозная обучаемая глубокая сеть, которая принимает изображение RAW в качестве входных данных и выполняет реконструкцию изображения, используя малый набор помеченных данных для адаптации исходного домена к целевому.

\subsubsection{Архитектура конвейера}\label{sect-2-1-1}

Архитектура рассматреваемого решения представляет собой AW-Net с уменьшеным по сравнению с оригиналом числом блоков повышающей и понижающей дискретизации, к которой были присоеденены дополнительные модули, необходимые для доменной адаптации конвейера, таие как: класификатор домена и преэнкодеры. Стоит также отметить, что подобная модификация может быть применима и к другим U-Net подобным ISP конвеерам, что, несомненно, значительно расширяет границы применимости рассматриваемого подхода. Рассмотрим каждый из вышеописанных блоков более детально.

\paragraph{Преэнкодер}

Преэнкодер представляет собой сверточную нейронную сеть, состоящую из трех слоев \textit{Convolution2D} с ядрами размером 3x3 и количеством фильтров, равным 8, 16 и 32 соответственно. В качестве активации используется \textit{ReLU}. Полная архитектура приведена в таблице \ref{pre-encoder-arch}. Данный модуль принимает в качестве входных данных не обработанное 4-канальное изображение GRGB RAW (до дебайеризации) размерности $H/2\times W/2\times 4$ и производит 32-канальный вывод размерности $H/2\times W/2\times 32$, который подается в общую AW-Net.

\begin{table}[H]
    \caption{Архитектура преэнкодера}\label{pre-encoder-arch}
    \begin{tabular}{|p{5cm}|p{3cm}|p{3cm}|}
        \hline
        {Слой} & {Ядро} & {Число фильтров} \\
        \hline
        Convolution2D & 3 & 8 \\
        \hline
        ReLU & - & - \\
        \hline
        Convolution2D & 3 & 16 \\
        \hline
        ReLU & - & - \\
        \hline
        Convolution2D & 3 & 32 \\
        \hline
        ReLU & - & - \\
        \hline
    \end{tabular}
\end{table}

Применение автоэнкодеров обусловлено огромным разрывом между различными устройствами в веду их уникальности. Цель преэнкодера заключается в уменьшении доменного разрыва между разными камерами путем извлечения индивидуальных и независимых признаков из каждой камеры. Иными словами, преэнкодер необходим для учета индивидуальных особенностей сенсора, таких как black level, индивидуальные шумы и т.д.

Во время доменной адаптации преэнкодеры обучаются независимо и подключаются напрямую к обучаемой нейронной сети, обучаясь вместе с основной моделью как единое целое. % Преэнкодер
\paragraph{Общая AW-Net}

AW-Net представляет собой модель автоэнкодера, отнсящегося к семейству U-Net подобных. Нейронная сеть содержит три блока понижающей дискретизации и четыре блока повышающей дискретизации. Данный блок был получен из оригинальной модели AW-Net путем сокращения количества соответствующих блоков понижения и повышения размерности с 5 до 3 и с 6 до 4 соответственно, а также изменения размерности входных данных для возможности включения преэнкодеров.

Изображение, полученное от преэнкодера, состоит из 32 каналов, и является входными данными для AW-Net. В результате обработки, AW-Net производит два выхода: 3-канальное изображение RGB размерности $H\times W\times 3$ и 256-мерный вектор узкого места, полученный путем применения к соответвующей карте признака глобального усреднения (\textit{GlobalAveragePooling2D}).

AW-Net предназначена для обучения полному конвейеру обработки изображений (ISP), чтобы воспроизвести результирующее изображение, а также для выделения доменно-зависимых признаков в 256-мерный вектор. Этот вектор используется классификатором домена для определения особенностей конкретной камеры и настройки параметров обработки изображения инвариантным к этим особенностям. 
 % AW-Net
\paragraph{Классификатор домена}

Для уменьшения расхождений между различными доменами и увеличения производительности, в работе был применен бинарный классификатор доменов \cite{ganin2016domain}, использующий обратное распространение ошибки \cite{ganin2015unsupervised}. Классификатор состоит из двух полносвязных слоев \textit{Dense} и принимает на вход 256-мерный вектор, полученный от AW-Net. Полная архитектура приведена в таблице \ref{domain-classifyer}. На выходе он возвращает число, предсказывающее принадлежность входных данных к тому или иному домену, находящееся в интервале от 0 до 1.

\begin{table}[H]
    \caption{Архитектура классификатора домена}\label{domain-classifyer}
    \begin{tabular}{|p{5cm}|p{5cm}|}
        \hline
        {Слой} & {Выходная размерность} \\
        \hline
        GradientReversal & - \\
        \hline
        Dense & 256 \\
        \hline
        ReLU & - \\
        \hline
        BatchNormalization & - \\
        \hline
        Dense & 1 \\
        \hline
        Sigmoid & - \\
        \hline
    \end{tabular}
\end{table}

Классификатор предметной области обучен предсказывать домен входных данных, тогда как главная задача сети заключается в предсказании выходных данных. В процессе обучения экстрактор признаков подвергается обратному распространению ошибки, что заставляет его создавать признаки, не зависящие от предметной области.

Такой подход позволяет общей AW-Net выделять доменно-независимые признаки, в то время как преэнкодеры могут аккумулировать индивидуальные особенности каждого сенсора.
 % AW-Net
 % Архитектура конвейера
\subsubsection{Алгоритм обучения}\label{sect-2-1-2}

Для того чтобы обеспечить успешную доменную адаптацию конвейера обработки изображений, предлагаетсяпровести обучение сети в несколько этапов, которые помогут добиться наилучших результатов.

\begin{enumerate}
    \item Первый этап заключается в предварительном обучении конвейера в исходном домене. На данном этапе используется только прекодер исходного домена и выход RGB AW-Net. При этом выходы классификатора домена не учитываются, а преэнкодер целевого домена не используется. При выполнении данного этапа конвейер обработки изображений обучается на большом наборе данных как если бы это был end-2-end конвейер без доменной адаптации.
    \item Второй этап заключается в инициализации целевого преэнкодера весами из исходного преэнкодера. Это позволяет начать обучение сети на основе уже настроенных параметров, что может значительно ускорить процесс обучения.
    \item Третий этап представляет собой этап адаптации домена, где производится обучение всй сети, используя весь исходный домен и небольшую часть целевого домена (10 изображений). На этом этапе на каждом шаге обучения производится последовательная подача оответвующих пар изображений RAW-RGB из целевого и исходного доменов в соответствующий преэнкодер и вычисляются соответствующие функции потерь. При этом учитываются предсказания классификатора предметной области и его обратный градиент.
\end{enumerate}

Такой подход позволяет обеспечить эффективную адаптацию домена, учитывая индивидуальные особенности каждого домена и выделяя доменно-инвариатные признаки. Кроме того, предварительное обучение и инициализация целевого преэнкодера позволяют снизить вычислительную сложность и ускорить процесс обучения а также значительно сократить число размеченных данных, что является важным преимуществом данного подхода. Стоит также отметить, как показано в эксериментах, что используя предлагаемый подход, удается достичь сопоставимого с конвейором обученным с нуля на полном наборе данных качества формирования изображений.

 % Алгоритм обучения
\subsubsection{Функции потерь}\label{sect-2-1-3}

В данном разделе приводится описание функций потерь, которые применяются на этапах предварительного обучения и доменной адаптации в рамках данного исследования. Следует отметить, что все описываемые функции потерь основаны на комбинации функции потерь восприятия \cite{lib-vgg-loss}, функции потерь $L_1$, функции потерь MS-SSIM \cite{lib-msssim}, а также функций потерь цвета и слияния экспозиции \cite{lib-exp-fusion}, которые будут более подробно описаны ниже. Для обозначения прогнозируемого изображения, здесь и далее используется символ $\hat{I}$, а истинное RGB-изображение обозначается как $I$.

\paragraph{Функция потерь восприятия}

Функция потерь восприятия (Perceptual loss), также известная как VGG loss -- это функция потерь, которая использует предобученную на крупномосшабном наборе данных сверточную нейронную сеть, такую как VGG-16 или VGG-19, для оценки качества генерируемых изображений. Она основывается на предположении, что высокоуровневые признаки, извлекаемые из изображений нейронной сетью, лучше коррелируют с восприятием человеком, чем низкоуровневые признаки, такие как средние значения пикселей или сумма квадратов разностей между пикселями, так как подразумевается что предобученная на большом обемеданных нейронная сеть обладает хорошей обобщающей способностью.

Эта функция потерь определяется как сумма средних квадратичных отклонений между выходами сверточной сети на исходном и сгенерированном изображениях:

\begin{equation}
    \label{eq:2-1-3-1}
    L^{vgg} = L_{2}(VGG(I)-VGG(\hat{I})),
\end{equation}

где $VGG$ -- вывод последнего сверточного слоя VGG.

Функция потерь может быть применена к разным слоям сверточной сети, в зависимости от того, насколько высокоуровневые признаки нужно учитывать при оценке качества изображений (рис. \ref{vgg-loss}). 

\addimghere{2-1-vgg-loss}{0.6}{Визуализация скрытых представлений функции потерь восприятия.}{vgg-loss}

Перцептуальная функция часто используется в генеративных моделях, таких как генеративно-состязательные сети (GAN), чтобы улучшить качество генерируемых изображений и сделать их более реалистичными. Она также может быть использована в других задачах компьютерного зрения, таких как восстановление изображений, суперразрешение, стилизация изображений и т.д.
\paragraph{$L_1$ норма}

В данной работе мы применяем $L_1$ норму как функцию потерь, так как она позволяет надежно контролировать оптимизацию значений пикселей в процессе обучения сети. Эта потеря определяется как сумма абсолютных разностей между предсказанным изображением $\hat{I}$ и истинным RGB-изображением $I$. Она является чувствительной к малым различиям между соответствующими пикселями и может эффективно сглаживать текстуры и убирать шум, за счет чего, применение данной функции потерь довольно популярно при обучении нейросетевых конвейеров обработки изображений.

Кроме того, мы не используем эту потерю при доменной адаптации. Это делается для того, чтобы избежать переобучения нейронной сети на конкретные данные и обеспечить ее способность к обобщению на новые данные, так как размер размеченой выбрки целевого домена мал.

\paragraph{Функция потерь MS-SSIM}

SSIM — это метрика качества изображения, которая оценивает сходство между оригинальным и восстановленным изображением. Она основывается на вычислении трех показателей: яркости, контрастности и структуры.

Многомасштабный SSIM ($MSSSIM$) расширяет эту идею и вычисляет структурное подобие на нескольких масштабах изображения. Он вычисляет сходство в яркости, контрастности и структуре на каждом масштабе, а затем объединяет результаты в один показатель, усредняя их.

В контексте данного исследования многомасштабная потеря структурного подобия используется для улучшения качества реконструированных RGB изображений. Чем меньше значение MSSSIM, тем больше различий между восстановленным и оригинальным изображениями, и тем выше значение функции потерь $L^{MSSSIM}$. Формально функцию потерь можно определить как:

\begin{equation}
    \label{eq:2-1-3-3}
    L^{MSSSIM} = 1 - MSSSIM(I, \hat{I}).
\end{equation}

В результате, нейронная сеть стремится минимизировать потери и максимизировать значение MSSSIM для получения более точного реконструированного изображения.
\paragraph{Функция цыетовых потерь}

В рамках данного метода также применяется функция потерь цвета $L^{rgb}$, которая измеряется как косинусное расстояние между RGB векторами соответсвующих изображений. Эта функция помогает уменьшить разницу в цветовых характеристиках между предсказанным изображением и оригинальным изображением, что способствует улучшению цветопередачи. Формально, потерю цвета можно определить следующим образом:

\begin{equation}
    \label{eq:2-1-3-4}
    L^{rgb} = 1 - \frac{\hat{I} \cdot I}{\left\|\hat{I}\right\|_2 \left\|I\right\|_2}.
\end{equation}

Косинусное расстояние вычисляется между векторами RGB, представленными в виде многомерных массивов. Наряду с $L_1$ нормой, применение косинусного расстояния в качестве фукции потерь также является распространенной практикой при обучении конвейеров обработки изображений.

\paragraph{Функция потерь слияния экспозиции}

Слияние экспозиции - это метод, объединения нескольких изображений с разными настройками экспозиции в одно изображение с высоким динамическим диапазоном (HDR), которое имеет хорошую общую экспозицию и сохранение деталей, предложенный Томом Мертенсом в 2007 году.

Техника экспонирования включает три основных этапа:

\begin{itemize}
    \item Создание стека экспозиции. Первым шагом является захват нескольких изображений одной и той же сцены с различными настройками экспозиции, такими как выдержка или диафрагма. Затем эти изображения выравниваются и объединяются в один стек.

    \item Создание карты весов. Для получения взвешеной карты для каждого пикселя в стеке изображений требуется выполнить ряд шагов. Сначала, необходимо вычислить три карты для каждого пикселя во всех изображениях в стеке. Первая карта представляет собой карту контраста, которая получается путем вычисления абсолютного значения отклика фильтра Лапласа, примененного к черно-белой версии каждого изображения в стеке. Фильтр Лапласа является эффективным инструментом для выявления краев и текстурных элементов в изображении. Вторая карта представляет собой карту насыщенности и вычисляется как стандартное отклонение по каналам RGB для каждого пикселя в изображениях. Наконец, для создания карты сбалансированности интенсивности, каждый канал изображения взвешивается с помощью кривой Гаусса в зависимости от отклонения его интенсивности от значения 0,5. Затем все результаты умножаются между собой. Полученная карта используются для вычисления весов каждого пикселя в стеке изображений \ref{exp-fusion}. 

    \item Слияние. Наконец, алгоритм слияния экспозиции смешивает изображения в стеке с использованием весов для создания окончательного HDR-изображения. 
\end{itemize}

\addimghere{2-1-3-exp-fusion}{0.7}{Изображения с разной выдержкой и соответствующие им карты весов.}{exp-fusion}

Метод слияния экспозиции имеет несколько преимуществ по сравнению с другими методами слияния HDR. Он эффективен с вычислительной точки зрения и создает естественные изображения с хорошей общей экспозицией и сохранением деталей.

Функция потерь слияния экспозиции сводит к минимуму разницу карты весов между предсказанным и реальным изображением, что помогает построить более точную экспозицию, и обеспечить наилучшую цветопередачу. Данную функцию можно определить как:

\begin{equation}
    \label{eq:2-1-3-5}
    L^{exp} = L_1(Exp(I), Exp(\hat{I})),
\end{equation}

где $Exp(X)$ -- карта весов полученная методом слияния экспозиций.

\paragraph{Этап предобучения}

На этапе предварительного обучения используется функция потерь, которая включает следующие компоненты: $L_1$ для минимизации разницы между предсказанными и реальными изображениями, $L^{vgg}$ для сохранения структурных свойств изображений, $L^{MSSSIM}$ для улучшения реконструированных изображений по индексу структурного подобия, $L^{rgb}$ для минимизации разницы в цвете между предсказанным изображением и реальным изображением, и $L^{exp}$ для сохранения хорошей экспозиции. Функция потерь определяется следующим образом:

\begin{equation}
    \label{eq:2-1-3-5}
    L^{pretrain} = L_1 + L^{vgg} + L^{MSSSIM} + L^{rgb} + L^{exp}.
\end{equation}

\paragraph{Этап доменной адаптации}

На данном этапе производится адаптация. При этом используюется малое количество данных (около десяти изображений) исходного и целевого домена, а также учитываются предсказания классификатора домена. При этом минимизируются три функции потерь:
\begin{itemize}
    \item Функция потерь исходного домена: $L^{source}=L_1$;
    \item Функция потерь целевого домена: $L^{target}=L^{vgg} + L^{MSSSIM} + L^{rgb} + L^{exp}$;
    \item Функция потерь классификатора доменов: $L^{classifier}=BCE$ (бинарная кросс-энтропия).
\end{itemize}

 % Losses