\paragraph{Предобучение}

В начальной фазе эксперимента было выполнено предварительное обучение нейросетевого конвейера на данных исходного домена. В этот период использовался только прекодер исходного домена, а также RGB выход AW-Net.

Нейросеть подвергалась процессу обучения с нуля на основе трех наборов данных: Zurich RAW-to-RGB, Mobile AIM21 и Samsung S7 ISP. Этот процесс обучения продолжался в течение четырех эпох.

В результате этого стадии обучения, нейросетевой конвейер достиг определенных показателей качества, которые представлены в таблице \ref{tab:from-zoro}. Эти результаты служат отправной точкой для следующих этапов исследования.

\begin{table}[H]
    \caption{Результаты обучения AW-Net с нуля.}\label{tab:from-zoro}
    \begin{tabular}{|p{5cm}|p{3cm}|p{3cm}|}
        \hline
        {Набор данных} & {PSNR} & {SSIM} \\
        \hline
        Zurich RAW-to-RGB & 19.46 & 0.73 \\
        \hline
        Mobile AIM21 & 23.48 & 0.87 \\
        \hline
        Samsung S7 ISP & 22.16 & 0.81 \\
        \hline
    \end{tabular}
\end{table}