\paragraph{Функция потерь MS-SSIM}

SSIM — это метрика качества изображения, которая оценивает сходство между оригинальным и восстановленным изображением. Она основывается на вычислении трех показателей: яркости, контрастности и структуры.

Многомасштабный SSIM ($MSSSIM$) расширяет эту идею и вычисляет структурное подобие на нескольких масштабах изображения. Он вычисляет сходство в яркости, контрастности и структуре на каждом масштабе, а затем объединяет результаты в один показатель, усредняя их.

В контексте данного исследования многомасштабная потеря структурного подобия используется для улучшения качества реконструированных RGB изображений. Чем меньше значение MSSSIM, тем больше различий между восстановленным и оригинальным изображениями, и тем выше значение функции потерь $L^{MSSSIM}$. Формально функцию потерь можно определить как:

\begin{equation}
    \label{eq:2-1-3-3}
    L^{MSSSIM} = 1 - MSSSIM(I, \hat{I}).
\end{equation}

В результате, нейронная сеть стремится минимизировать потери и максимизировать значение MSSSIM для получения более точного реконструированного изображения.