\paragraph{Шумоподавление}

\addimghere{1-1-2-denoising}{0.9}{Примеры шумоподавления. Слева: шумное изображение. В центре: шумоподавление с использованием bm3d. Справа: шумоподавление с использованием NLM.}{denoising}

Алгоритмы шумоподавления \cite{lib-denoising} представляют собой важный этап в улучшении качества изображения. При удалении шума необходимо сохранять баланс между его устранением и сохранением мелких деталей изображения (рис. \ref{denoising}). Чрезмерно агрессивное шумоподавление может привести к потере четкости изображения, а недостаточное – к сохранению визуального шума, что может отвлекать внимание от окончательного результата. Алгоритмы шумоподавления учитывают различные факторы, такие как уровень усиления ISO и экспозиция. В зависимости от ISP конвейера шумоподавление модет применяться после разных его шагов, как после преобразования цветового пространства, так и после дебайеринга. Ниже приведены некоторые из наиболее распространенных алгоритмов шумоподавления:

\begin{itemize}
    \item Пространственные фильтры \cite{lib-spatial-filters}: это простые фильтры, которые вычисляют среднее значение значений пикселей в небольшой окрестности, чтобы получить отфильтрованное значение пикселей. Примеры включают фильтр усреднения, медианный фильтр и фильтр Гаусса \cite{lib-gaussian-filter}.

    \item Фильтр нелокальных средств (Non-local Means, NLM) \cite{lib-nlm}: фильтр NLM представляет собой фильтр на основе исправлений, который использует избыточность изображения для удаления шума. Он работает, сравнивая сходство между участками изображения, чтобы определить веса для фильтрации.

    \item Двусторонний фильтр (Bilateral Filter) \cite{lib-bilateral}. Двусторонний фильтр — это нелинейный фильтр, сохраняющий края изображения и удаляющий шум. Он использует фильтр Гаусса для взвешивания значений пикселей на основе их пространственного расстояния и фильтр диапазона для взвешивания значений пикселей на основе их расстояния интенсивности.
    \item BM3D (Block-Matching and 3D Filtering) — популярный алгоритм шумоподавления изображений на основе патчей, который использует самоподобие и избыточность в естественных изображениях для эффективного удаления шума при сохранении деталей и текстур изображения \cite{lib-bm3d}. Алгоритм работает в два основных этапа: совместная фильтрация и фильтрация Винера. На первом этапе алгоритм группирует похожие участки изображения, выполняя сопоставление блоков. Это достигается разделением изображения на перекрывающиеся блоки и поиском похожих блоков в окне поиска вокруг каждого блока. Подобные патчи затем складываются для формирования массива 3D-данных. На втором этапе сложенные патчи фильтруются с помощью коллаборативного фильтра Винера. Фильтр предназначен для использования сходства между сложенными патчами для улучшения отношения сигнал/шум. Коэффициенты фильтра оцениваются с использованием байесовской структуры, где априорная информация о распределении участков изображения используется для упорядочения фильтра. Одной из основных сильных сторон алгоритма BM3D является его способность обрабатывать различные типы шума, включая гауссовский, пуассоновский и импульсный шум. 
\end{itemize}
