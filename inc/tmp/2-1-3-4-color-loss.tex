\paragraph{Функция цыетовых потерь}

В рамках данного метода также применяется функция потерь цвета $L^{rgb}$, которая измеряется как косинусное расстояние между RGB векторами соответсвующих изображений. Эта функция помогает уменьшить разницу в цветовых характеристиках между предсказанным изображением и оригинальным изображением, что способствует улучшению цветопередачи. Формально, потерю цвета можно определить следующим образом:

\begin{equation}
    \label{eq:2-1-3-4}
    L^{rgb} = 1 - \frac{\hat{I} \cdot I}{\left\|\hat{I}\right\|_2 \left\|I\right\|_2}.
\end{equation}

Косинусное расстояние вычисляется между векторами RGB, представленными в виде многомерных массивов. Наряду с $L_1$ нормой, применение косинусного расстояния в качестве фукции потерь также является распространенной практикой при обучении конвейеров обработки изображений.
