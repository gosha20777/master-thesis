\paragraph{Баланс белого}

Баланс белого выполняется для имитации способности зрительной системы человека выполнять хроматическую адаптацию к освещению сцены. Баланс белого часто называют вычислительной константностью цвета, чтобы обозначить связь со зрительной системой человека. Баланс белого
требует оценки отклика цветового фильтра R, G, B сенсора на освещение сцены. Этот отклик можно предварительно откалибровать на заводе, записав отклик датчика на спектры обычного освещения (например, солнечного света, ламп накаливания и флуоресцентного освещения). Эти предварительно откалиброванные настройки затем становятся частью предустановки баланса белого камеры, которую пользователь может выбрать. Более распространенная альтернатива — полагаться на алгоритм автоматического баланса белого (AWB) камеры, который оценивает реакцию датчика R, G, B на освещение непосредственно из захваченного изображения (рис. \ref{white-balance}).

\addimghere{1-2-1-white-balance}{0.8}{Принцип работы алгоритма баланса белого}{white-balance}


Одним из самых распространенных AWB алгоритмов является алгоритм серого мира. Алгоритм серого мира -- это простой алгоритм постоянства цвета, который предполагает, что в среднем цвета в изображении должны иметь одинаковые средние значения по каналам красного, зеленого и синего цветов \cite{lib-gray-world}. Другими словами, предполагается, что снимаемая сцена имеет нейтральный цветовой баланс. Данный подход работает, оценивая средний цвет изображения и используя эту оценку для масштабирования отдельных цветовых каналов, чтобы они имели равные средние значения.