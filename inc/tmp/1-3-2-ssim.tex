\subsubsection{SSIM}\label{sect-1-3-2}

SSIM измеряет сходство между двумя изображениями с точки зрения яркости, контрастности и структурной информации (рис. \ref{ssim}). В отличие от PSNR, SSIM предназначен для отражения человеческого восприятия качества изображения, и было показано, что он хорошо коррелирует с субъективными оценками качества изображения. В частности, SSIM сравнивает структурное сходство между соответствующими пикселями на двух изображениях, принимая во внимание яркость и контрастность, а также пространственные отношения между пикселями.

\begin{equation}
    \label{eq:1-3-2}
    SSIM(x,y) = \frac{(2\mu_x\mu_y + C_1)(2\sigma_{xy} + C_2)}{(\mu_x^2 + \mu_y^2 + C_1)(\sigma_x^2 + \sigma_y^2 + C_2)}
\end{equation}

где $x$ и $y$ — два изображения, $\mu_x$ и $\mu_y$ — средние значения пикселей изображений, $\sigma_x^2$ и $\sigma_y^2$ — дисперсии пикселей изображений, $\sigma_{xy}$ — ковариация пикселей изображений, $C_1$ и $C_2$ — константы, используемые для стабилизации деления.

\addimghere{1-3-2-ssim}{0.5}{Визуализация работы SSIM.}{ssim}

Значения SSIM находятся в диапазоне от 0 до 1, причем более высокие значения указывают на большее сходство между двумя изображениями. Оценка SSIM, равная 1, указывает на то, что два изображения идентичны. SSIM стал популярной метрикой для оценки качества изображения и видео, особенно в тех случаях, когда человеческое восприятие является важным фактором при оценке.
