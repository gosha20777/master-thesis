\subsubsection{Классические конвейеры обработки изображений}\label{sect-1-2-1}

Конвейер обработки изображений представляет собой последовательность шагов обработки изображения в результате которого исходное RAW изображение преобразуется в цветное трех канальное изображение. Конвейеры обработки изображений используются во всех современных цифровых фотоаппаратах и мобильных устройствах. Системы камер
иметь специальный чип, называемый процессором сигналов изображен (ISP), который выполняет эту обработку конвейера за миллисекунды для каждого изображения. Следует отметить, что эти конвейеры сильно различаются в зависимости от производителя камеры и часто являются проприетарными, что затрудняет исследование.  Классические конвейеры обработки сигналов были впервые проанализированы М. Брауном \cite{lib-borwn}, который выделил следующй ряд шагов преобразования изображения, таких как, нормировка уровня черного, демозаикинг, шумоподавление, коррекцию баланса белого, преобразование цветового пространства, гамма-коррекцию и др. Эти шаги можно применять либо к одному кадру, либо к нескольким кадрам, при этом для последних требуются дополнительные шаги, такие как выравнивание изображения (рис. \ref{classic-isp}). Ниже приведем краткое описание основных шагов конвейера обработки изображений.

\addimghere{1-2-1-classic-isp}{0.9}{Классические конвейеры обработки изображений для одного (сверху) и нескольких кадров (снизу)}{classic-isp}

\paragraph{Предобработка}

На этапе предобработки необработанный сигнал датчика камеры нормализуется путем ограничения его значений в диапазоне от 0 до 1. Однако, в связи с ошибками датчика, минимальное значение часто смещено от 0 в большую сторону на некоторую велечену, называемую BlackLevel. Этот параметр обычно зависит от различных параметров камеры, таких как ISO и усиление. Параметр WhiteLevel определяет максимальное значение. Данные параметры используются для нормализации изображения между 0 и 1. При предварительной обработке также производится исправление дефектных пикселей на датчике. Маска дефектных пикселей калибруется на заводе и отображает места с неисправными фотодиодами. Дефектные пиксели могут быть фотодиодами, которые всегда сообщают о высоком значении (горячий пиксель), или которые не выводят никакого значения (битый пиксель). Значения дефектных пикселей интерполируются с использованием их соседей. Наконец, применяется коррекция затенения объектива (цветового веньетирования), чтобы исправить эффекты неравномерного попадания света на датчик. Маска затенения объектива предварительно калибруется производителем и корректируется для каждого кадра в соответствии с различными уровнями яркости, коэффициентами усиления и предполагаемым освещением сцены, используемым для баланса белого. % Предобработка
\paragraph{Алгоритм демозаики}

Алгоритмы демозаики \cite{lib-demosaic} применяется для преобразования одноканального необработанного изображения в трехканальное полноразмерное изображение RGB. Наиболее простым алгоритмом является метод билинейной интерполяции, который берет среднее значение значений цвета соседних пикселей для оценки отсутствующих значений цвета (рис \ref{demosaicing}). Этот метод прост в реализации и эффективен в вычислительном отношении, но может привести к заметным цветовым артефактам, таким как эффект Муара, и отсутствию деталей в результирующем изображении по сравнению с более продвинутыми методами демозаики.

\addimghere{1-2-1-demosaicing}{0.8}{Иллюстрирует общего подхода демозаики. Показан красный пиксель и соседние с ним байеровские пиксели. Необходимо оценить недостающие значения зеленого и синего пикселей, которые могут быть найдены  путем интерполяции. Весовая маска, основанная на сходстве красного пикселя с его соседями, вычисляется для управления этой интерполяцией. Эта взвешенная интерполяция помогает избежать размытия по краям сцены.}{demosaicing}

Алгоритм демозаики Малвара является более продвинутым по сравнению с билинейной демозаикой, и он известен созданием высококачественных изображений с уменьшенными цветовыми артефактами и повышенной детализацией. Данный подход основан на идее оценки отсутствующей цветовой информации для каждого пикселя на основе градиента цветовой информации окружающих его пикселей. Чтобы улучшить качество билинейного метода, в алгоритм были добавлены лапласовские межканальные поправки. Одной из ключевых особенностей алгоритма демозаики Малвара является то, что он учитывает пространственно-частотное содержание изображения, различая высокочастотные и низкочастотные шумы, что позволяет алгоритму лучше сохранять мелкие детали и края изображения, уменьшая при этом цветовые артефакты (рис. \ref{demosaicing-errors}).

\addimghere{1-1-2-bayer-errors}{0.8}{Дефекты билинейного алгоритма демозаики}{demosaicing-errors}

Стоит отметить, что большинство алгоритмов демозаики представляют собой проприетарные методы, которые часто также выполняют отсечение светлых участков, повышение резкости и некоторое начальное шумоподавление.  % Демозаикинг
\paragraph{Баланс белого}

Баланс белого выполняется для имитации способности зрительной системы человека выполнять хроматическую адаптацию к освещению сцены. Баланс белого часто называют вычислительной константностью цвета, чтобы обозначить связь со зрительной системой человека. Баланс белого
требует оценки отклика цветового фильтра R, G, B сенсора на освещение сцены. Этот отклик можно предварительно откалибровать на заводе, записав отклик датчика на спектры обычного освещения (например, солнечного света, ламп накаливания и флуоресцентного освещения). Эти предварительно откалиброванные настройки затем становятся частью предустановки баланса белого камеры, которую пользователь может выбрать. Более распространенная альтернатива — полагаться на алгоритм автоматического баланса белого (AWB) камеры, который оценивает реакцию датчика R, G, B на освещение непосредственно из захваченного изображения (рис. \ref{white-balance}).

\addimghere{1-2-1-white-balance}{0.8}{Принцип работы алгоритма баланса белого}{white-balance}


Одним из самых распространенных AWB алгоритмов является алгоритм серого мира. Алгоритм серого мира -- это простой алгоритм постоянства цвета, который предполагает, что в среднем цвета в изображении должны иметь одинаковые средние значения по каналам красного, зеленого и синего цветов \cite{lib-gray-world}. Другими словами, предполагается, что снимаемая сцена имеет нейтральный цветовой баланс. Данный подход работает, оценивая средний цвет изображения и используя эту оценку для масштабирования отдельных цветовых каналов, чтобы они имели равные средние значения. % Баланс белого
\paragraph{Преобраование цветового пространства}

После применения баланса белого изображение по-прежнему находится в цветовом пространстве RGB, связанным с камерой. Этап преобразования цветового пространства выполняется для преобразования изображения из цветового пространства необработанного RGB датчика в независимое от устройства воспринимаемое цветовое пространство, полученное непосредственно из цветового пространства CIE 1931 XYZ \cite{lib-cie}. В большинстве камер используется цветовое пространство ProPhoto RGB с широкой гаммой, способеное отображать 90\% цветов, видимых обычному наблюдателю. Самым распространенным спомобом нахождения преобразования цветового пространства является нахождение наилучшего линейного преобразования между набором цветов из цветового пространства связаого с камерой и набором цветов из цветового пространства CIE XYZ \cite{lib-brown-cst}. Это преобразование может быть представлено в виде матрицы $3 \times 3$ (рис. \ref{cst}).

\addimghere{1-2-1-cst}{0.4}{Преобраование цветового пространства}{cst} % Преобразование цветового пространства
\paragraph{Шумоподавление}

\addimghere{1-1-2-denoising}{0.9}{Примеры шумоподавления. Слева: шумное изображение. В центре: шумоподавление с использованием bm3d. Справа: шумоподавление с использованием NLM.}{denoising}

Алгоритмы шумоподавления \cite{lib-denoising} представляют собой важный этап в улучшении качества изображения. При удалении шума необходимо сохранять баланс между его устранением и сохранением мелких деталей изображения (рис. \ref{denoising}). Чрезмерно агрессивное шумоподавление может привести к потере четкости изображения, а недостаточное – к сохранению визуального шума, что может отвлекать внимание от окончательного результата. Алгоритмы шумоподавления учитывают различные факторы, такие как уровень усиления ISO и экспозиция. В зависимости от ISP конвейера шумоподавление модет применяться после разных его шагов, как после преобразования цветового пространства, так и после дебайеринга. Ниже приведены некоторые из наиболее распространенных алгоритмов шумоподавления:

\begin{itemize}
    \item Пространственные фильтры \cite{lib-spatial-filters}: это простые фильтры, которые вычисляют среднее значение значений пикселей в небольшой окрестности, чтобы получить отфильтрованное значение пикселей. Примеры включают фильтр усреднения, медианный фильтр и фильтр Гаусса \cite{lib-gaussian-filter}.

    \item Фильтр нелокальных средств (Non-local Means, NLM) \cite{lib-nlm}: фильтр NLM представляет собой фильтр на основе исправлений, который использует избыточность изображения для удаления шума. Он работает, сравнивая сходство между участками изображения, чтобы определить веса для фильтрации.

    \item Двусторонний фильтр (Bilateral Filter) \cite{lib-bilateral}. Двусторонний фильтр — это нелинейный фильтр, сохраняющий края изображения и удаляющий шум. Он использует фильтр Гаусса для взвешивания значений пикселей на основе их пространственного расстояния и фильтр диапазона для взвешивания значений пикселей на основе их расстояния интенсивности.
    \item BM3D (Block-Matching and 3D Filtering) — популярный алгоритм шумоподавления изображений на основе патчей, который использует самоподобие и избыточность в естественных изображениях для эффективного удаления шума при сохранении деталей и текстур изображения \cite{lib-bm3d}. Алгоритм работает в два основных этапа: совместная фильтрация и фильтрация Винера. На первом этапе алгоритм группирует похожие участки изображения, выполняя сопоставление блоков. Это достигается разделением изображения на перекрывающиеся блоки и поиском похожих блоков в окне поиска вокруг каждого блока. Подобные патчи затем складываются для формирования массива 3D-данных. На втором этапе сложенные патчи фильтруются с помощью коллаборативного фильтра Винера. Фильтр предназначен для использования сходства между сложенными патчами для улучшения отношения сигнал/шум. Коэффициенты фильтра оцениваются с использованием байесовской структуры, где априорная информация о распределении участков изображения используется для упорядочения фильтра. Одной из основных сильных сторон алгоритма BM3D является его способность обрабатывать различные типы шума, включая гауссовский, пуассоновский и импульсный шум. 
\end{itemize}
 % Шумоподавление
