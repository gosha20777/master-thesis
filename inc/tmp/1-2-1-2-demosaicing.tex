\paragraph{Алгоритм демозаики}

Алгоритмы демозаики \cite{lib-demosaic} применяется для преобразования одноканального необработанного изображения в трехканальное полноразмерное изображение RGB. Наиболее простым алгоритмом является метод билинейной интерполяции, который берет среднее значение значений цвета соседних пикселей для оценки отсутствующих значений цвета (рис \ref{demosaicing}). Этот метод прост в реализации и эффективен в вычислительном отношении, но может привести к заметным цветовым артефактам, таким как эффект Муара, и отсутствию деталей в результирующем изображении по сравнению с более продвинутыми методами демозаики.

\addimghere{1-2-1-demosaicing}{0.8}{Иллюстрирует общего подхода демозаики. Показан красный пиксель и соседние с ним байеровские пиксели. Необходимо оценить недостающие значения зеленого и синего пикселей, которые могут быть найдены  путем интерполяции. Весовая маска, основанная на сходстве красного пикселя с его соседями, вычисляется для управления этой интерполяцией. Эта взвешенная интерполяция помогает избежать размытия по краям сцены.}{demosaicing}

Алгоритм демозаики Малвара является более продвинутым по сравнению с билинейной демозаикой, и он известен созданием высококачественных изображений с уменьшенными цветовыми артефактами и повышенной детализацией. Данный подход основан на идее оценки отсутствующей цветовой информации для каждого пикселя на основе градиента цветовой информации окружающих его пикселей. Чтобы улучшить качество билинейного метода, в алгоритм были добавлены лапласовские межканальные поправки. Одной из ключевых особенностей алгоритма демозаики Малвара является то, что он учитывает пространственно-частотное содержание изображения, различая высокочастотные и низкочастотные шумы, что позволяет алгоритму лучше сохранять мелкие детали и края изображения, уменьшая при этом цветовые артефакты (рис. \ref{demosaicing-errors}).

\addimghere{1-1-2-bayer-errors}{0.8}{Дефекты билинейного алгоритма демозаики}{demosaicing-errors}

Стоит отметить, что большинство алгоритмов демозаики представляют собой проприетарные методы, которые часто также выполняют отсечение светлых участков, повышение резкости и некоторое начальное шумоподавление. 