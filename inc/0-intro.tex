\anonsection{Введение}

В настоящее время технологии машинного обучения и искусственного интеллекта имеют огромную значимость, так как они значительно упрощают обработку больших объемов данных, позволяют предсказывать события, автоматизировать производственные процессы, делать научные открытия и решать другие задачи \cite{lib-sidorenko}. Одной из важных областей машинного обучения является компьютерное зрение, которое использует сверточные нейронные сети для обработки изображений и видео и делает различные выводы на их основе. Цифровая фотография также является необходимой для применения компьютерного зрения.

Вычислительная фотография является областью компьютерного зрения, которая использует вычислительные методы для улучшения качества изображений, полученных с помощью фото- и видеокамер. Глубокие сверточные нейронные сети достигли значительного прогресса в компьютерном зрении, включая идентификацию объектов, классификацию и сегментацию изображений. Однако сбор крупномасштабных наборов данных для каждого нового сенсора устройства является трудоемким и дорогостоящим процессом.

В настоящее время качество камер мобильных телефонов играет важную роль в смартфонах, поэтому большее внимание уделяется конвейеру фрорирования изображений (агл. Image Signal Processing Piepline -- ISP). Цель научного сообщества заключается в разработке сквозного конвейера на основе нейронных сетей, чтобы исключить дорогостоящий и утомительный процесс настройки ISP конвейера для каждого нового устройства. Однако основным недостатком нейронного подхода является необходимость подготовки большого набора данных при разработке нового смартфона.

\addimghere{0-abstract.pdf}{0.9}{Визуализация различных подходов к обучению ISP конвейеров. Предлагаемый подход к адаптации домена с несколькими кадрами эффективен в задаче формирования изображения. Заесь $S$ -- это набор данных Zurich RAW-to-RGB, а $T$ -- Mobile AIM21.}{abstract}

В данной работе предлагается новый метод быстрой адаптации существующего нейросетевого конвейера к новому домену, который требует всего 10 помеченных изображений целевого домена для достижения самой современной производительности на наборах данных реальных тестов камеры (рис. \ref{abstract}). Это позволит значительно снизить стоимость производства ISP конвейеров на основе нейронных сетей для каждого нового устройства.

В разделе \ref{sect-1-1} описана проблема формирования цветного изображения, физическая модель формирования изображения, принцип работы цифровай камеры, подчеркивается проблема уникальности каждой цифровой камеры. В разделе \ref{sect-1-2} приводится принцип работы классического и нейросетового конвейеров обработки изображений, описаны современные методы анализа изображений. В разделе \ref{sect-1-3} говорится о метриках оценивания ISP конвейеров. В разделе \ref{sect-1-4} приводится постановка задачи ддоменной адаптации и дается описание метода распространения обратного градиента. В разделе \ref{sect-2-1} описывается предложенный метод доменной адаптации нейросетевого конвейера Aw-Net. В разделе \ref{sect-2-2} приводятся результаты экспериментов.

\clearpage