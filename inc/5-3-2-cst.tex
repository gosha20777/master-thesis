\subsubsection{Нахождение матрицы перехода}

В ходе реализации стратегии доменной адаптации в качестве отправной точки была применена методика полиномиальной линейной регрессии для определения матрицы перехода между цветовыми пространствами Samsung S7 ISP и Mobile AIM21 во всех возможных комбинациях. Для выполнения этого процесса был использован специфический набор данных, описанный в разделе \ref{sec:calibration-dataset} данного исследования.

Эксперимент включал в себя несколько ключевых этапов. На первом этапе производилось обучение модели полиномиальной линейной регрессии, с целью определения матрицы перехода между цветовыми пространствами размерностью 3x3. Это позволило сформировать взаимосвязи между элементами исходных и целевых доменов. Затем, уже выявленная матрица перехода применялась к тестовой выборке, составленной на основе данных Mobile AIM21 и Samsung S7 ISP. Таким образом было осуществлено приведение данных из Mobile AIM21 в Samsung S7 ISP и наоборот. Наконец, обновленный и преобразованный набор данных использовался для подачи в модели AW-Net, которые были заранее обучены на соответствующих наборах данных: Samsung S7 ISP и Mobile AIM21. Это позволило проверить эффективность использования найденной матрицы перехода для преобразования доменов.

Полученные результаты применения обученной матрицы перехода представлены в сравнительной таблице \ref{tab:cst}.

\begin{table}[H]
    \caption{Результаты дообучения AW-Net. Метрики качества на тестовой выборке (PSNR, SSIM) дообучения AW-Net на различных наборах данных. На главной диагонали показаны метрики обучения с нуля.}\label{tab:cst}
    \begin{tabular}{|ll|ll|}
        \hline
        \multicolumn{2}{|l|}{\multirow{2}{*}{}}                               & \multicolumn{2}{c|}{Target Domain}                               \\ \cline{3-4} 
        \multicolumn{2}{|l|}{}                                                & \multicolumn{1}{l|}{Samsung S7 ISP}       & Mobile AIM21         \\ \hline
        \multicolumn{1}{|c|}{\multirow{2}{*}{Source Domain}} & Samsung S7 ISP & \multicolumn{1}{l|}{\textbf{22.16, 0.81}} & 19.85, 0.72          \\ \cline{2-4} 
        \multicolumn{1}{|c|}{}                               & Mobile AIM21   & \multicolumn{1}{l|}{18.96, 0.71}          & \textbf{23.48, 0.87} \\ \hline
    \end{tabular}
\end{table}

