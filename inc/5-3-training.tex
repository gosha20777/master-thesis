\subsection{Обучение моделей и результаты}\label{sect-5-3}

В рамках нашего эксперимента по доменной адаптации нейросетевого конвейера обработки изображений, мы применили все возможные комбинации вышеуказанных наборов данных. Начальная стадия включала предварительное обучение предложенной модификации AW-Net на полном наборе данных, после чего была выполнена доменная адаптация конвейера на выборке из 10 изображений.

Для оценки эффективности предложенного подхода, мы провели сравнительный анализ, включая дообучение конвейера на полном обучающем наборе. Важно отметить, что для получения более точных и надежных результатов, каждый эксперимент был выполнен пять раз, а полученные измерения качества были усреднены.

В процессе обучения использовался оптимизатор Adam с параметрами по умолчанию: скорость обучения ($lr$) равна 0.001, первый момент ($\beta1$) - 0.9, второй момент ($\beta1$) - 0.9, и $\epsilon$ - 1е-7. Результаты эксперимента представлены на рисунках \ref{vis-mai} и \ref{vis-zurich}, что позволяет наглядно увидеть эффективность предложенного подхода.

\subsubsection{Предобучение}

В начальной фазе эксперимента было выполнено предварительное обучение нейросетевого конвейера на данных исходного домена. В этот период использовался только прекодер исходного домена, а также RGB выход AW-Net.

Нейросеть подвергалась процессу обучения с нуля на основе трех наборов данных: Zurich RAW-to-RGB, Mobile AIM21 и Samsung S7 ISP. Этот процесс обучения продолжался в течение четырех эпох.

В результате этого стадии обучения, нейросетевой конвейер достиг определенных показателей качества, которые представлены в таблице \ref{tab:from-zoro}. Эти результаты служат отправной точкой для следующих этапов исследования.

\begin{table}[H]
    \caption{Результаты обучения AW-Net с нуля.}\label{tab:from-zoro}
    \begin{tabular}{|p{5cm}|p{3cm}|p{3cm}|}
        \hline
        {Набор данных} & {PSNR} & {SSIM} \\
        \hline
        Zurich RAW-to-RGB & 19.46 & 0.73 \\
        \hline
        Mobile AIM21 & 23.48 & 0.87 \\
        \hline
        Samsung S7 ISP & 22.16 & 0.81 \\
        \hline
    \end{tabular}
\end{table}
\subsubsection{Простое дообучение}

Для моделей, успешно обученных на предыдущем этапе, была применена стратегия простого дообучения (transfer learning) для соответствующих оставшихся наборов данных во всевозможных комбинациях. Этот процесс привел к генерации шести различных комбинаций данных.

В процессе дообучения был задействован весь объем обучающей выборки. Каждая из моделей подвергалась дообучению в течение двух эпох.

Результаты этого этапа дообучения представлены в сравнительной таблице \ref{tab:tl}. Эти данные отражают что применение стратегии transfer learning на 12\% уступает обучению с нуля.

\begin{table}[H]
    \caption{Результаты дообучения AW-Net. Метрики качества на тестовой выборке (PSNR, SSIM) дообучения AW-Net на различных наборах данных. На главной диагонали показаны етрики обучения с нуля.}\label{tab:tl}
    \begin{tabular}{|p{3cm}p{3cm}|p{3cm}p{3cm}p{3cm}|}
        \hline
        \multicolumn{2}{|p{3cm}|}{\multirow{2}{*}{}}                     & \multicolumn{3}{p{5cm}|}{Целевой домен}                                                                                \\ \cline{3-5} 
        \multicolumn{2}{|p{3cm}|}{}                                      & \multicolumn{1}{p{3cm}|}{Zurich RAW-to-RGB}    & \multicolumn{1}{p{3cm}|}{Mobile AIM21}         & Samsung S7 ISP       \\ \hline
        \multicolumn{1}{|p{3cm}|}{Исходный домен}    & Zurich RAW-to-RGB & \multicolumn{1}{p{3cm}|}{\textbf{19.46, 0.73}} & \multicolumn{1}{p{3cm}|}{22.04, 0.79}          & 20.25, 0.74          \\ \cline{2-5} 
        \multicolumn{1}{|p{3cm}|}{}                  & Mobile AIM21      & \multicolumn{1}{p{3cm}|}{16.97, 0.68}          & \multicolumn{1}{p{3cm}|}{\textbf{23.48, 0.87}} & 20.15, 0.74          \\ \cline{2-5} 
        \multicolumn{1}{|p{3cm}|}{}                  & Samsung S7 ISP    & \multicolumn{1}{p{3cm}|}{16.74, 0.67}          & \multicolumn{1}{p{3cm}|}{21.89, 0.78}          & \textbf{22.16, 0.81} \\ \hline
    \end{tabular}
\end{table}


\subsubsection{Доменная адаптация на нескольких изображениях}

\addimghere{2-2-3-zurich-da}{0.9}{Визуализация результатов дменной адаптации из Zurich RAW-to-RGB в Mobile AIM19 и Samsung S7 ISP, а также сравнение с дообучением и обучением с нуля.}{vis-zurich}

Процесс адаптации домена был выполнен, применяя аналогичные комбинации наборов данных, как и в случае дообучения. Отметим, что ключевым элементом предложенного подхода было использование лишь десяти изображений в полном разрешении для доменной адаптации, при этом обучение нейросети продолжалось в течение двух эпох.

Результаты данного этапа представлены в таблице \ref{tab:da} и иллюстрированы на рисунках \ref{vis-mai} и \ref{vis-zurich}. 

\begin{table}[H]
    \caption{Результаты доменной адаптации AW-Net на дксяти изображениях. Приведены метрики качества на тестовой выборке (PSNR, SSIM). На главной диагонали показаны етрики обучения с нуля.}\label{tab:da}
    \begin{tabular}{|p{3cm}p{3cm}|p{3cm}p{3cm}p{3cm}|}
        \hline
        \multicolumn{2}{|p{3cm}|}{\multirow{2}{*}{}}                     & \multicolumn{3}{p{5cm}|}{Целевой домен}                                                                          \\ \cline{3-5} 
        \multicolumn{2}{|p{3cm}|}{}                                      & \multicolumn{1}{p{3cm}|}{Zurich RAW-to-RGB}  & \multicolumn{1}{p{3cm}|}{Mobile AIM21}       & Samsung S7 ISP     \\ \hline
        \multicolumn{1}{|p{3cm}|}{Исходный домен}    & Zurich RAW-to-RGB & \multicolumn{1}{p{3cm}|}{\textbf{19.46, 0.73}} & \multicolumn{1}{p{3cm}|}{23.15, 0.86}        & 22.07, 0.79        \\ \cline{2-5} 
        \multicolumn{1}{|p{3cm}|}{}                  & Mobile AIM21      & \multicolumn{1}{p{3cm}|}{18.92, 0.71}        & \multicolumn{1}{p{3cm}|}{\textbf{23.48, 0.87}} & 22.03, 0.79        \\ \cline{2-5} 
        \multicolumn{1}{|p{3cm}|}{}                  & Samsung S7 ISP    & \multicolumn{1}{p{3cm}|}{18.85, 0.71}        & \multicolumn{1}{p{3cm}|}{23.08, 0.85}        & \textbf{22.16, 0.81} \\ \hline
        \end{tabular}
\end{table}

Как можно заметить, предложенный подход к адаптации домена позволяет достигнуть высокой эффективности. Результаты превосходят дообучение на полном наборе данных на 14\% и лишь на 2\% уступают результатам обучения с нуля, несмотря на значительное уменьшение объема обучающей выборки.

\addimghere{2-2-3-mai-da}{0.9}{Визуализация результатов доменной адаптации из Mobile AIM19 в Zurich RAW-to-RGB и Samsung S7 ISP, а также сравнение с дообучением и обучением с нуля.}{vis-mai}

Дополнительно, была проведена серия экспериментов с использованием 1, 5 и 10 изображений (рис. \ref{vis-k-da}) для адаптации домена от Zurich RAW-to-RGB к Mobile AIM21, результаты которых представлены в таблице \ref{tab:k-da}. 

\begin{table}[H]
    \caption{Результаты обучения AW-Net с нуля.}\label{tab:k-da}
    \begin{tabular}{|p{3cm}|p{3cm}|p{3cm}|}
        \hline
        {Число изображений $k$} & {PSNR ↑} & {SSIM ↑} \\
        \hline
        1 & 19.05 & 0.77 \\
        \hline
        5 & 21.70 & 0.79 \\
        \hline
        10 & 23.15 & 0.86  \\
        \hline
    \end{tabular}
\end{table}

Из полученных результатов видно, что наш подход позволяет достичь качества, близкого к уровню современных передовых методов (SotA), с падением качества всего на 2\%, используя не более 10 образцов. С использованием 5 изображений падение качества составляет 8\%, что все равно превосходит результаты дообучения. При использовании одного изображения падение составляет 19\%.

\addimghere{2-2-3-k-da}{0.9}{Визуализация результатов адаптации домена из Zurich RAW-to-RGB в Mobile AIM21 с использованием изображений $k = 1, 5, 10$.}{vis-k-da}
