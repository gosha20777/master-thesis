\paragraph{Обратноый градиент}

Domain-Adversarial Training -- это метод доменной адаптации с обратным градиентом, предложенный Ярославом Ганиным и Виктором Лемпицким в их статье 2016 года \cite{lib-reverse-gradient}.

\addimghere{1-4-1-reverse-gradient}{0.9}{Обратный градиент.}{reverse-gradient}

Основная идея предметно-состязательного обучения заключается в использовании классификатора доменов для различения исходного и целевого доменов при обучении нейронной сети выполнению конкретной задачи (рис. \ref{reverse-gradient}). Классификатор предметной области обучен предсказывать предметную область входных данных, в то время как основная задача сети -- предсказание выходных данных. Во время обучения к экстрактору признаков применяется обратный градиент, что заставляет его создавать признаки, не зависящие от предметной области. Затем классификатор предметной области обучается минимизировать ошибку классификации, а экстрактор признаков обучается максимизировать ошибку классификации предметной области. Этот процесс помогает согласовать распределения функций между исходным и целевым доменами. Обратный градиент используется для обновления экстрактора признаков таким образом, что классификатор домена затрудняет различие между исходным и целевым доменами. 