\paragraph{AW-Net}

\addimghere{1-2-2-awnet}{0.9}{Архитектура AW-Net.}{awnet}

Предлагаемая AWNet использует структуру, напоминающую U-Net, и объединяет архитектуру с помощью трех основных модулей, а именно модуля глобальной плотности контекста и модулей повышающей и понижающей дискретизации остаточного вейвлета. Модуль глобального контекста состоит из остаточного полносвзного блока (Residual Dense Block, RDB) и глобального блока контекста (Glocal Context Block, GCB) (рис. \ref{awnet-gc}). Данные блоки остаточной информации используются для улучшения отображения цветов. Всего в RDB используется семь сверточных слоев, где первые шесть слоев нацелены на увеличение количества карт объектов, а последний слой объединяет все карты объектов, сгенерированные из этих слоев. В конце RDB представлен блок глобального контекста, чтобы побудить сеть изучить глобальное сопоставление цветов, поскольку локальное сопоставление цветов может привести к ухудшению результатов из-за несовпадения пикселей между парами изображений RAW и RGB. 

\addimghere{1-2-2-awnet-gc}{0.9}{Glocal Context Block.}{awnet-gc}

Для повышающей и понижающей дискретизации используется идея идею дискретного вейвлет-преобразования (Discrete Wavelet Transform, DWT), что позволяет разлагать карты входных признаков на высокочастотную и низкочастотную составляющие, которые подключаются к блоку повышающей дискретизации для восстановления изображения. Стоит отметить, что карты признаков, созданные с помощью операции в частотной области, могут не иметь пространственной корреляции. Поэтому используется дополнительный пространственный сверточный слой для понижения дискретизации карты объектов. Комбинация операций в частотной и пространственной областях облегчает изучение множества особенностей таких как цветокоррекция и шумоподавление.

В процессе обработки изображений в конвейере были задействованы две модели AW-Net, которые были обучены независимо друг от друга. На вход моделей были поданы одноканальные и трехканальные изображения, полученные путем чтения необработанного паттерна Байера и изображения, полученного после демозаики. Для получения итогового результата восстановленные изображения были усреднены.
