\anonsection{Заключение}

В рамках данной работы представлен подход к адаптации домена для конвейера обработки изображений RAW-to-RGB, основанный на использовании ограниченного числа размеченных выборок из целевой области и значительного объема выборок из исходной области. Эффективность такого подхода особенно актуальна для производителей цифровых камер и смартфонов, ведь он позволяет сократить как финансовые, так и временные затраты на разработку новых устройств.

В процессе реализации предложенной методологии сначала проводится сбор информации о камере посредством преэнкодеров. Затем происходит извлечение инвариантных к домену характеристик с помощью сети AW-Net. Инвариантность призноков достигается в следствии применения подхода к адаптации домена с использованием обратного градиента для уменьшения разрыва между доменами.

Результаты исследования демонстрируют, что в сравнении с методами, основанными на обучении с использованием обширных данных целевой области, применение предложенного подхода даже с незначительным числом размеченных образцов из целевой области (около десятка) обеспечивает сравнимый уровень качества, уступая обучению на полном наборе данных всего на 2\%.

Уверенность в том, что представленный подход будет стимулировать дальнейшие исследования в данной области и будет применяться в процессе производства цифровых камер, основывается на полученных в ходе исследования результатах.

\clearpage