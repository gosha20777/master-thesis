\subsubsection{PSNR}\label{sect-1-3-1}

PSNR измеряет разницу между двумя изображениями с точки зрения среднеквадратичной ошибки (Mean Squared Error, MSE) между значениями их пикселей (рис. \ref{psnr}). В частности, PSNR определяется как отношение максимально возможной мощности сигнала к мощности искажающего шума, влияющего на достоверность его представления:

\begin{equation}
    \label{eq:1-3-1}
    PSNR = 10 \cdot \log_{10} \left( \frac{MAX_I^2}{MSE} \right)
\end{equation}


где $MAX_I$ — максимальное значение пикселя изображения, а $MSE$ — среднеквадратичная ошибка между двумя изображениями. Более высокие значения PSNR указывают на более высокий уровень сходства между исходным и сжатым изображением, при этом значения выше 30 дБ обычно считаются хорошим качеством.

\addimghere{1-3-1-psnr}{0.5}{Визуализация работы PSNR.}{psnr}

Хотя PSNR широко используется и его легко вычислить, он имеет ограничения в своей способности фиксировать воспринимаемое качество изображения. Это связано с тем, что он не учитывает пространственную информацию и контраст между пикселями изображения. 

