\subsubsection{Общая AW-Net}

AW-Net представляет собой модель автоэнкодера, относящегося к семейству U-Net подобных. Нейронная сеть содержит три блока понижающей дискретизации и четыре блока повышающей дискретизации. Данный блок был получен из оригинальной модели AW-Net путем сокращения количества соответствующих блоков понижения и повышения размерности с 5 до 3 и с 6 до 4 соответственно, а также изменения размерности входных данных для возможности включения преэнкодеров.

Изображение, полученное от преэнкодера, состоит из 32 каналов, и является входными данными для AW-Net. В результате обработки, AW-Net производит два выхода: 3-канальное изображение RGB размерности $H\times W\times 3$ и 256-мерный вектор узкого места, полученный путем применения к соответствующей карте признака глобального усреднения (\textit{GlobalAveragePooling2D}).

AW-Net предназначена для обучения полному конвейеру обработки изображений (ISP), чтобы воспроизвести результирующее изображение, а также для выделения доменно-зависимых признаков в 256-мерный вектор. Этот вектор используется классификатором домена для определения особенностей конкретной камеры и настройки параметров обработки изображения инвариантным к этим особенностям. 
