\subsubsection{Классификатор домена}

Для уменьшения расхождений между различными доменами и увеличения производительности, в работе был применен бинарный классификатор доменов \cite{ganin2016domain}, использующий обратное распространение ошибки \cite{ganin2015unsupervised}. Классификатор состоит из двух полносвязных слоев \textit{Dense} и принимает на вход 256-мерный вектор, полученный от AW-Net. Полная архитектура приведена в таблице \ref{domain-classifyer}. На выходе он возвращает число, предсказывающее принадлежность входных данных к тому или иному домену, находящееся в интервале от 0 до 1.

\begin{table}[H]
    \caption{Архитектура классификатора домена}\label{domain-classifyer}
    \begin{tabular}{|p{5cm}|p{5cm}|}
        \hline
        {Слой} & {Выходная размерность} \\
        \hline
        GradientReversal & - \\
        \hline
        Dense & 256 \\
        \hline
        ReLU & - \\
        \hline
        BatchNormalization & - \\
        \hline
        Dense & 1 \\
        \hline
        Sigmoid & - \\
        \hline
    \end{tabular}
\end{table}

Классификатор предметной области обучен предсказывать домен входных данных, тогда как главная задача сети заключается в предсказании выходных данных. В процессе обучения экстрактор признаков подвергается обратному распространению ошибки, что заставляет его создавать признаки, не зависящие от предметной области.

Такой подход позволяет общей AW-Net выделять доменно-независимые признаки, в то время как преэнкодеры могут аккумулировать индивидуальные особенности каждого сенсора.
