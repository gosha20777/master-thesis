\subsubsection{Предобработка}

На этапе предобработки необработанный сигнал датчика камеры нормализуется путем ограничения его значений в диапазоне от 0 до 1. Однако, в связи с ошибками датчика, минимальное значение часто смещено от 0 в большую сторону на некоторую величину, называемую BlackLevel. Этот параметр обычно зависит от различных параметров камеры, таких как ISO и усиление. Параметр WhiteLevel определяет максимальное значение. Данные параметры используются для нормализации изображения между 0 и 1. При предварительной обработке также производится исправление дефектных пикселей на датчике. Маска дефектных пикселей калибруется на заводе и отображает места с неисправными фотодиодами. Дефектные пиксели могут быть фотодиодами, которые всегда сообщают о высоком значении (горячий пиксель), или которые не выводят никакого значения (битый пиксель). Значения дефектных пикселей интерполируются с использованием их соседей. Наконец, применяется коррекция затенения объектива (цветового веньетирования), чтобы исправить эффекты неравномерного попадания света на датчик. Маска затенения объектива предварительно калибруется производителем и корректируется для каждого кадра в соответствии с различными уровнями яркости, коэффициентами усиления и предполагаемым освещением сцены, используемым для баланса белого.