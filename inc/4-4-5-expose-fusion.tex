\subsubsection{Функция потерь слияния экспозиции}

Слияние экспозиции - это метод, объединения нескольких изображений с разными настройками экспозиции в одно изображение с высоким динамическим диапазоном (HDR), которое имеет хорошую общую экспозицию и сохранение деталей, предложенный Томом Мертенсом в 2007 году.

Техника экспонирования включает три основных этапа:

\begin{itemize}
    \item Создание стека экспозиции. Первым шагом является захват нескольких изображений одной и той же сцены с различными настройками экспозиции, такими как выдержка или диафрагма. Затем эти изображения выравниваются и объединяются в один стек.

    \item Создание карты весов. Для получения взвешенной карты для каждого пикселя в стеке изображений требуется выполнить ряд шагов. Сначала, необходимо вычислить три карты для каждого пикселя во всех изображениях в стеке. Первая карта представляет собой карту контраста, которая получается путем вычисления абсолютного значения отклика фильтра Лапласа, примененного к черно-белой версии каждого изображения в стеке. Фильтр Лапласа является эффективным инструментом для выявления краев и текстурных элементов в изображении. Вторая карта представляет собой карту насыщенности и вычисляется как стандартное отклонение по каналам RGB для каждого пикселя в изображениях. Наконец, для создания карты сбалансированности интенсивности, каждый канал изображения взвешивается с помощью кривой Гаусса в зависимости от отклонения его интенсивности от значения 0,5. Затем все результаты умножаются между собой. Полученная карта используются для вычисления весов каждого пикселя в стеке изображений \ref{exp-fusion}. 

    \item Слияние. Наконец, алгоритм слияния экспозиции смешивает изображения в стеке с использованием весов для создания окончательного HDR-изображения. 
\end{itemize}

\addimghere{2-1-3-exp-fusion}{0.7}{Изображения с разной выдержкой и соответствующие им карты весов.}{exp-fusion}

Метод слияния экспозиции имеет несколько преимуществ по сравнению с другими методами слияния HDR. Он эффективен с вычислительной точки зрения и создает естественные изображения с хорошей общей экспозицией и сохранением деталей.

Функция потерь слияния экспозиции сводит к минимуму разницу карты весов между предсказанным и реальным изображением, что помогает построить более точную экспозицию, и обеспечить наилучшую цветопередачу. Данную функцию можно определить как:

\begin{equation}
    \label{eq:5-3-5}
    L^{exp} = L_1(Exp(I), Exp(\hat{I})),
\end{equation}

где $Exp(X)$ -- карта весов полученная методом слияния экспозиций.
