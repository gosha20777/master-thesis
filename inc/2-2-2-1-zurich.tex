\paragraph{Zurich RAW-to-RGB}

Набор данных, который был представлен в 2021 году для участия в конкурсе Mobile AIM19, на момент текущей даты, является самым обширным в своем роде. Он включает более 20 000 синхронных пар уличных изображений, полученных с использованием цифровой зеркальной камеры Canon 5D Mark IV и смартфона Huawei P20, оснащенного мобильным сенсором Sony IMX380 с разрешением 12 мегапикселей. Huawei P20 захватывает изображения в формате RAW, тогда как Canon 5D Mark IV служет для генерации GRB изображений по средтвам его проприетарного ISP (рис. \ref{zurich}).

\addimghere{2-2-2-zurich}{0.9}{Примеры изображений из набора данных Zurich RAW-to-RGB. Слева направо: исходное 4-канальное изображение RAW (каналы = [R, GR, B, GB])
сохранено в формате PNG, исходное визуализированное изображение RAW и целевое изображение Canon 5D Mark IV.}{zurich}

Фотографии были сделаны в течение дня в множестве различных мест, и в условиях сильно варьирующегося освещения и погодных условий. Важно отметить, что между парами изображений RAW и RGB существует некоторое отличие, поскольку они были получены на разных камерах. Для минимизации этих отличий, изображения были разделены на участки, и из них были отобраны наиболее схожие с помощью анализа ключевых точек с использованием алгоритма SIFT. Также было проведено выравнивание пар изображений для дальнейшей работы с ними.

Учитывая ограниченность возможностей обучения моделей глубокого обучения на изображениях высокого разрешения, было решено извлечь участки размером 224×224 пикселя из предварительно совмещенных пар изображений Huawei P20/Canon. В результате этого процесса было отобрано 48043 участка, что приблизительно равно 60 областям на одно полноразмерное изображение.