\subsection{Физическая модель формирования изображения}\label{sect-1-1}

Физическая модель формирования изображения \cite{lib-maximov} описывает процесс, который происходит в природе при формировании изображения на сенсоре (камере) или на глазном яблоке (рис. \ref{linear-model}).

\addimghere{1-1-1-linear-model}{0.6}{Линейная модель формирования изображения}{linear-model}

При обсуждении цветовой теории необходимо уделить внимание физической стороне вопроса. Из физической точки зрения, свет может быть представлен в виде электромагнитной волны с длиной $\lambda$. Представим, что на сцену падает свет от источника, который освещает объект, а зрительная система человека или камеры может детектировать световой сигнал. Источники света создают на поверхности объектов освещенность, которая описывается пространственным распределением спектральной освещенности $S(\lambda)$. Когда свет падает на объект, часть света отражается, а часть поглощается, что придает объекту окраску, которая описывается спектром поглощения $\Phi(\lambda)$. Когда световой сигнал попадает на зрительную систему человека или камеры, он описывается пространственным распределением спектральной чувствительности $\chi(\lambda)$. Световой сигнал $I(\lambda)$, который воспринимает человек или камера, может быть описан следующим образом:

\begin{equation}
\label{eq:1-1-1}
I(\lambda) = \int_{0}^{\infty} S(\lambda) \Phi(\lambda) \chi(\lambda) d\lambda
\end{equation}.

Нетрудно показать, в зависимости от выбранной кривой спектральной чувствительности $\chi(\lambda)$, световой сигнал $I(\lambda)$ может быть представлен в виде трехмерного вектора, который называется цветовым вектором.
