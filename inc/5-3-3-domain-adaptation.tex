\subsubsection{Доменная адаптация на нескольких изображениях}

\addimghere{2-2-3-zurich-da}{0.9}{Визуализация результатов дменной адаптации из Zurich RAW-to-RGB в Mobile AIM19 и Samsung S7 ISP, а также сравнение с дообучением и обучением с нуля.}{vis-zurich}

Процесс адаптации домена был выполнен, применяя аналогичные комбинации наборов данных, как и в случае дообучения. Отметим, что ключевым элементом предложенного подхода было использование лишь десяти изображений в полном разрешении для доменной адаптации, при этом обучение нейросети продолжалось в течение двух эпох.

Результаты данного этапа представлены в таблице \ref{tab:da} и иллюстрированы на рисунках \ref{vis-mai} и \ref{vis-zurich}. 

\begin{table}[H]
    \caption{Результаты доменной адаптации AW-Net на десяти изображениях. Приведены метрики качества на тестовой выборке (PSNR, SSIM). На главной диагонали показаны метрики обучения с нуля.}\label{tab:da}
    \begin{tabular}{|p{3cm}p{3cm}|p{3cm}p{3cm}p{3cm}|}
        \hline
        \multicolumn{2}{|p{3cm}|}{\multirow{2}{*}{}}                     & \multicolumn{3}{p{5cm}|}{Целевой домен}                                                                          \\ \cline{3-5} 
        \multicolumn{2}{|p{3cm}|}{}                                      & \multicolumn{1}{p{3cm}|}{Zurich RAW-to-RGB}  & \multicolumn{1}{p{3cm}|}{Mobile AIM21}       & Samsung S7 ISP     \\ \hline
        \multicolumn{1}{|p{3cm}|}{Исходный домен}    & Zurich RAW-to-RGB & \multicolumn{1}{p{3cm}|}{\textbf{19.46, 0.73}} & \multicolumn{1}{p{3cm}|}{23.15, 0.86}        & 22.07, 0.79        \\ \cline{2-5} 
        \multicolumn{1}{|p{3cm}|}{}                  & Mobile AIM21      & \multicolumn{1}{p{3cm}|}{18.92, 0.71}        & \multicolumn{1}{p{3cm}|}{\textbf{23.48, 0.87}} & 22.03, 0.79        \\ \cline{2-5} 
        \multicolumn{1}{|p{3cm}|}{}                  & Samsung S7 ISP    & \multicolumn{1}{p{3cm}|}{18.85, 0.71}        & \multicolumn{1}{p{3cm}|}{23.08, 0.85}        & \textbf{22.16, 0.81} \\ \hline
        \end{tabular}
\end{table}

Как можно заметить, предложенный подход к адаптации домена позволяет достигнуть высокой эффективности. Результаты превосходят дообучение на полном наборе данных на 14\% и лишь на 2\% уступают результатам обучения с нуля, несмотря на значительное уменьшение объема обучающей выборки.

\addimghere{2-2-3-mai-da}{0.9}{Визуализация результатов доменной адаптации из Mobile AIM19 в Zurich RAW-to-RGB и Samsung S7 ISP, а также сравнение с дообучением и обучением с нуля.}{vis-mai}

Дополнительно, была проведена серия экспериментов с использованием 1, 5 и 10 изображений (рис. \ref{vis-k-da}) для адаптации домена от Zurich RAW-to-RGB к Mobile AIM21, результаты которых представлены в таблице \ref{tab:k-da}. 

\begin{table}[H]
    \caption{Результаты обучения AW-Net с нуля.}\label{tab:k-da}
    \begin{tabular}{|p{3cm}|p{3cm}|p{3cm}|}
        \hline
        {Число изображений $k$} & {PSNR ↑} & {SSIM ↑} \\
        \hline
        1 & 19.05 & 0.77 \\
        \hline
        5 & 21.70 & 0.79 \\
        \hline
        10 & 23.15 & 0.86  \\
        \hline
    \end{tabular}
\end{table}

Из полученных результатов видно, что наш подход позволяет достичь качества, близкого к уровню современных передовых методов (SotA), с падением качества всего на 2\%, используя не более 10 образцов. С использованием 5 изображений падение качества составляет 8\%, что все равно превосходит результаты дообучения. При использовании одного изображения падение составляет 19\%.

\addimghere{2-2-3-k-da}{0.9}{Визуализация результатов адаптации домена из Zurich RAW-to-RGB в Mobile AIM21 с использованием изображений $k = 1, 5, 10$.}{vis-k-da}
