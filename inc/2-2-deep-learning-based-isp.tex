\subsection{Нейросетевые конвейеры обработки изображений}\label{sect-2-2}

С ростом производительности смартфонов, конвейеры обработки сигналов изображения (ISP) на основе глубокого обучения, стали перспективным подходом к улучшению качества изображения цифровых камер. Эти конвейеры используют мощь алгоритмов машинного обучения для изучения преобразования необработанных данных датчика в окончательное изображение RGB. Это позволяет конвейеру изучить оптимальное сопоставление данных датчика с конечным изображением, что может привести к лучшему качеству изображения и снижению шума. В то время как традиционные конвейеры состоят из ряда предопределенных этапов обработки изображения, нейросетевые конвейеры могут быть полностью сквозными. 

Одним из ключевых преимуществ обученных ISP является их способность адаптироваться к различным условиям без изменения архитектуры. Это связано с тем, что конвейер может научиться учитывать специфические шумовые и цветовые характеристики конкретного сенсора камеры и быть оптимизирован для конкретных задач визуализации, таких как фотосъемка при слабом освещении или визуализация с высоким динамическим диапазоном (HDR).

Использование машинного обучения для преобразования изображений RAW в RGB также привело к значительным улучшениям в вычислительной фотографии и поспособствовало улучшению различных шагов классических конвейеров. Например, камера Google Pixel использует обученный конвейер ISP, который сочетает традиционные методы обработки изображений с алгоритмами машинного обучения для получения высококачественных изображений в различных условиях освещения.

Также в последнее время набирают популярность одношаговые, сквозные (end-2-end) конвейеры для вычислительной фотографии, где весь конвейер обработки изображений обучается использованием больших наборов данных парных изображений RAW и RGB. Такие конвейеры позволяют избежать проблемы сопоставления между этапами обработки изображений, которая может привести к нежелательным артефактам, а также зачастую обладают более высокой производительностью и имеют значительно меньше гипер-параметров, что облегчает их калибровку и настройку. Однако, такие конвейеры требуют большого количества данных для обучения, что может быть проблемой для некоторых приложений.


\addimghere{1-2-2-unet}{0.9}{U-Net архитектура.}{unet}

Стоит отметить, что практически все современные нейросетевые end-2-end конвейеры обработки изображений базируются на U-Net подобной архитектуре \cite{lib-u-net}, представляющей из себя сверточную нейронную сеть, которая состоит из энкодера и декодера со сквозными связями (skip connections) (рис. \ref{unet}). Такая архитектура позволяет извлекать признаки из изображения и восстанавливать его до исходного размера.

Первый сквозной конвейер был предложен в 2018 году \cite{lib-deep-isp}, вместе с первым же набором данных Samsung s7 ISP Dataset. В дальнейшем количество открытых наборов данных увеличилось \cite{lib-zurich-raw-rgb}, что спровоцировало интерес научного сообщества к таким конвейерам. Так в 2019 году был представлен конвейер PyNET \cite{lib-py-net}, и AW-Net \cite{lib-aw-net} в 2020, ставший победителем соревнований Mobile AIM 2020 \cite{lib-mobile-aim-2020}. В последующие годы к числу таких архитектур прибавились dh\_isp и MW-ISPNet \cite{lib-aim21}.

\input{inc/2-2-1-aw-net} % AW-Net
